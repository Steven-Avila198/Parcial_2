\documentclass{article}
\usepackage[utf8]{inputenc}
\usepackage[spanish]{babel}
\usepackage{listings}
\usepackage{graphicx}
\graphicspath{ {images/} }
\usepackage{cite}
\begin{document}

\begin{titlepage}
    \begin{center}
        \vspace*{1cm}
            
        \Huge
        \textbf{PARCIAL 2}
            
        \vspace{0.5cm}
        \LARGE
        INFORME
            
        \vspace{1.5cm}
            
        \textbf{Michael Stiven Zapata Giraldo\\Brayan Steven Avila Marin}
            
        \vfill
            
        \vspace{0.8cm}
            
        \Large
        Despartamento de Ingeniería Electrónica y Telecomunicaciones\\
        Universidad de Antioquia\\
        Medellín\\
        Septiembre de 2021
            
    \end{center}
\end{titlepage}

\tableofcontents
\newpage
\section{ Analisis y diseño del problema }\label{intro}

a) Para este proyecto consideramos usar una matriz de leds 100x100 con tiras de 10 neopixels para la representación de las imagenes. Para programarla usaremos la librería Adafruit Neopixel. La conexión de la tira de leds es bastante sencilla, solo necesitaremos usar un solo pin del Arduino, y conectarlo al pin Din de la tira de leds, y en cada tira termina la conexion en la salida en el pin DO para en la siguiente conexion hacerla en la entrada de la siguiente tira y asi sucesivamente. Lo importante aca es tener en cuenta que cada led puede consumir hasta 60mA.

b)Esquema:

-Se define la cantidad de Led para hacer la matriz(100 leds con diez tiras)
-Se utiliza una fuente externa de energia para la alimentacion de las tiras de leds desde el arduino
-Se Implementa la clase dafruit Neopixel para crear efectos sobre las tiras de leds.

\begin{figure}[ht]
\includegraphics[width=10cm]{Esquema.PNG}
\centering
\caption{Esquema del montaje del circuito}
\label{fig:Montaje}
\end{figure}

c)Algoritmo

d)Consideraciones a tener en cuenta en la implementación:

-Conexiones de las tiras de led: entrada por el pin DIN y salidas por el pin DO.
-Establecer una fuente de energia o  fuente de voltaje de 5V/1A
-Utilizaremos la clase setPixelColor para apagar algunios leds y la clase show para actuaizar los leds

\section{ diseño de la solución planteada }



\section{Esquema donde describa las tareas que usted definió en el desarrollo del algoritmo.}


\section{Algoritmo implementado.}



\section{Evolución del algoritmo y consideraciones a tener en cuenta en la implementación.
}




\end{document}
